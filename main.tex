\documentclass[11pt]{article}

% NOTE: Only include the packages you need/use!
% The below is only a suggestion with some extras needed for examples.

% BASIC SETTINGS
%%%%%%%%%%%%%%%%%%%%%%%%%%%%%%%%%%%%%%%%%%%%%%%%%%%%%%%%%%%%%%%%%%%%%%%%%%%%%%%
\usepackage[english]{babel}         % Language settings
\usepackage[utf8]{inputenc}         % Input settings
\usepackage[T1]{fontenc}            % Output settings
\usepackage{lmodern}                % Latin modern font
\usepackage{textcomp}               % Fonts, symbols etc.
\usepackage{microtype}              % Improved micro-typography
\usepackage[top=3cm, bottom=3cm, inner=3cm, outer=3cm]{geometry}
\usepackage[dvipsnames]{xcolor}                 % Coloured text

%%%%%%%%%%%%%%%%%%%%%%%%%%%%%%%%%%%%%%%%%%%%%%%%%%%%%%%%%%%%%%%%%%%%%%%%%%%%%%%
% MATHS
\usepackage{amsmath}                % Mathematical expressions
\usepackage{amssymb}                % Mathematical symbols
\usepackage{mathtools}              % Moar maths, e.g. :=
\usepackage{commath}                % E.g. \abs{} och \norm{}
\usepackage{dsfont}                 % Double-stroke font, e.g. natural numbers
\usepackage{bm}                     % Bold math symbols
\usepackage{cancel}
\usepackage{braket}
\usepackage{diffcoeff}
% Use the correct ds in diffs
\difdef{f,s,c}{}{
  op-symbol = \mathrm{d},
}
\difdef{f,s}{f}{
	op-symbol = \delta,
}
% \usepackage{tensor}               % Index notation
% \usepackage{accents}              % Math accents
% \usepackage{braket}               % Dirac bra-ket notation

%%%%%%%%%%%%%%%%%%%%%%%%%%%%%%%%%%%%%%%%%%%%%%%%%%%%%%%%%%%%%%%%%%%%%%%%%%%%%%%
% FIGURES
\usepackage{graphicx}               % Figures
\usepackage{float}                  % Position enforcement using [H]
% \usepackage{pdfpages}               % Include PDF-pages

%%%%%%%%%%%%%%%%%%%%%%%%%%%%%%%%%%%%%%%%%%%%%%%%%%%%%%%%%%%%%%%%%%%%%%%%%%%%%%%
% TABLES
\usepackage{array}                  % (TODO)
\usepackage{tabularx}               % (TODO)
\usepackage{diagbox}                % Slash in tables
\usepackage{booktabs}               % Improved rules in tables

% Reduce weight of top and bottom rules
\setlength{\heavyrulewidth}{0.075em}

% TABLE LAYOUT (optional)
%\newcommand{\PreserveBackslash}[1]{\let\temp=\\#1\let\\=\temp}
%\newcolumntype{C}[1]{>{\PreserveBackslash\centering}p{#1}}
%\newcolumntype{R}[1]{>{\PreserveBackslash\raggedleft}p{#1}}
%\newcolumntype{L}[1]{>{\PreserveBackslash\raggedright}p{#1}}

%%%%%%%%%%%%%%%%%%%%%%%%%%%%%%%%%%%%%%%%%%%%%%%%%%%%%%%%%%%%%%%%%%%%%%%%%%%%%%%
% CODE LISTING

%%%%%%%%%%%%%%%%%%%%%%%%%%%%%%%%%%%%%%%%%%%%%%%%%%%%%%%%%%%%%%%%%%%%%%%%%%%%%%%
% TIKZ
\usepackage{tikz}                     % Tikz figures
\usetikzlibrary{fadings}
\usetikzlibrary{decorations.pathmorphing}

%%%%%%%%%%%%%%%%%%%%%%%%%%%%%%%%%%%%%%%%%%%%%%%%%%%%%%%%%%%%%%%%%%%%%%%%%%%%%%%
% CAPTION LAYOUT
\usepackage[
	labelfont       = {bf, it},
	textfont		= it,
    font            = normalsize,
    width           = 0.92\textwidth,
    justification   = justified,
    singlelinecheck = true
]{caption}
% NOTE: Consider matching the caption font size (normalsize above)
% to the font size in the figure/table. \captionsetup can be used
% to change settings locally.

% Caption for subfigures (optional)
\usepackage{subcaption}

% \DeclareCaptionLabelFormat{r-parens}{#2)} %Define our custom label
% \captionsetup[subfigure]{font=scriptsize, textfont=sl,%
% labelformat=r-parens, width=0.8\textwidth, position=b}

%%%%%%%%%%%%%%%%%%%%%%%%%%%%%%%%%%%%%%%%%%%%%%%%%%%%%%%%%%%%%%%%%%%%%%%%%%%%%%%
% CITATIONS/BIBLIOGRAPHY
\usepackage{silence}  % Suppress warnings (manually)
\WarningFilter{biblatex}{File 'english-ieee.lbx'}

\usepackage[
    backend     = biber,
    style       = ieee,
    dashed      = false,
    maxnames    = 6,
    natbib      = true,
    urldate     = iso,
    seconds     = true,
    isbn        = false
]{biblatex}

\DefineBibliographyStrings{english}{%
  mathesis = {MSc thesis},
}
\renewcommand{\subtitlepunct}{\addcolon\addspace}

\addbibresource{references.bib}

%%%%%%%%%%%%%%%%%%%%%%%%%%%%%%%%%%%%%%%%%%%%%%%%%%%%%%%%%%%%%%%%%%%%%%%%%%%%%%%
% SI UNITS
\usepackage{siunitx}

\sisetup{output-decimal-marker={.}}
\sisetup{exponent-product={\cdot}}
\sisetup{range-phrase=--}
\sisetup{per-mode=symbol}
% \sisetup{separate-uncertainty=true}
%\sisetup{round-mode=places,round-precision=3}
%\sisetup{parse-numbers = false}

%Speed of light and eV per speed of light
\DeclareSIUnit\clight{\text{\ensuremath{c}}}
\DeclareSIUnit\eVperc{\eV\per\clight}
\DeclareSIUnit\au{\text{a.u.}}

%%%%%%%%%%%%%%%%%%%%%%%%%%%%%%%%%%%%%%%%%%%%%%%%%%%%%%%%%%%%%%%%%%%%%%%%%%%%%%%
% HEADER & FOOTER LAYOUT
\usepackage{fancyhdr}
\usepackage{chappg}

\pagestyle{fancy}
\setlength{\headheight}{15pt}  % Increase head height

\renewcommand{\chaptermark}[1]{\markboth{\thechapter. \space#1}{\thechapter. \space#1}}
\renewcommand{\sectionmark}[1]{\markright{\thesection. \space#1}}

\if\thesisLayout 1 % One-sided
    \fancyhf{}
    \fancyhead[L]{\nouppercase{ \leftmark}}
    \fancyhead[R]{\nouppercase{ \rightmark}}
    \fancyfoot[C]{\thepage}
\fi

\if\thesisLayout 2 % Two-sided
      \fancyhf{}
    \fancyhead[LE]{\nouppercase{ \leftmark}}
    \fancyhead[RO]{\nouppercase{ \rightmark}}
    \fancyfoot[LE,RO]{\thepage}
    \fancypagestyle{plain}{% Redefine the plain page style
    \fancyhf{}
    \renewcommand{\headrulewidth}{0pt}
    \fancyfoot[LE,RO]{\thepage}}
\fi

%%%%%%%%%%%%%%%%%%%%%%%%%%%%%%%%%%%%%%%%%%%%%%%%%%%%%%%%%%%%%%%%%%%%%%%%%%%%%%%
% FOOTNOTE LAYOUT
\usepackage[bottom, hang]{footmisc}
% The multiple option does not seem to work. It should produce commas
% between markers in cases like "Lipsum\footnote{Foo}\footnote{Bar}".
% For now, use the following:
\newcommand{\footnotemarksep}{\textsuperscript{,}}
% A better approach might be: https://tex.stackexchange.com/a/62091

% Change the width of the ruler above footnotes:
\makeatletter
\renewcommand*{\footnoterule}{\kern-3\p@ \hrule \kern2.6\p@}
\makeatother

% Space before footnote text (0pt for marker width):
\setlength{\footnotemargin}{0pt}

% Might be preferable for multi-paragraph footnotes:
%\setlength{\footnotesep}{\baselineskip}

% For footnote symbols instead of numbers, use options symbol*
% and perpage with footmisc.
% To change symbol set, use e.g.: \setfnsymbol{wiley}

%%%%%%%%%%%%%%%%%%%%%%%%%%%%%%%%%%%%%%%%%%%%%%%%%%%%%%%%%%%%%%%%%%%%%%%%%%%%%%%
% BLANK LINE AND SPACING
\usepackage[raggedright]{titlesec}
\usepackage{parskip}                % Enables vertical spaces correctly

\setlength{\parindent}{0pt}
\setlength{\parskip}{\baselineskip}

% Title spacing

% Defaults:
%\titlespacing*{\chapter} {0pt}{50pt}{40pt}
%\titlespacing*{\section} {0pt}{3.5ex plus 1ex minus .2ex}{2.3ex plus .2ex}
%\titlespacing*{\subsection} {0pt}{3.25ex plus 1ex minus .2ex}{1.5ex plus .2ex}
%\titlespacing*{\subsubsection}{0pt}{3.25ex plus 1ex minus .2ex}{1.5ex plus .2ex}
%\titlespacing*{\paragraph} {0pt}{3.25ex plus 1ex minus .2ex}{1em}
%\titlespacing*{\subparagraph} {\parindent}{3.25ex plus 1ex minus .2ex}{1em}

\setlength{\belowdisplayskip}{0pt}
\setlength{\belowdisplayshortskip}{0pt}
\setlength{\abovedisplayskip}{0pt}
\setlength{\abovedisplayshortskip}{0pt}

\raggedbottom % does not fill the entire page if not necessary

% Line spacing:
% NOTE: setspace must be loaded before footmisc if both are used!
%\usepackage{setspace}              % (TODO)
%\setstretch{1.2}
%\linespread{1.2}

% Modify tolerances
%\pretolerance=1100
%\tolerance=8000
%\emergencystretch=0pt
%\righthyphenmin=4
%\lefthyphenmin=4

%%%%%%%%%%%%%%%%%%%%%%%%%%%%%%%%%%%%%%%%%%%%%%%%%%%%%%%%%%%%%%%%%%%%%%%%%%%%%%%
% TITLE AND TOC LAYOUT
\usepackage{titletoc}
\usepackage[title]{appendix}

% Define the number of section levels to be included in toc and numbered
\setcounter{tocdepth}{3}
\setcounter{secnumdepth}{3}

% Chapter styles (NOTE: only the last one is used)
\newcommand*{\thesisChapterStyle}{1} % 0 for default

\ifnum\thesisChapterStyle=1
    \titleformat{\chapter}[hang]{\fontsize{30}{10}\selectfont}
    {{\fontsize{30pt}{1em}\vspace{-5.2ex}\selectfont \textnormal{\thechapter. \hspace{1pt}}}}
    {.5ex}{\raggedright}[\rule{\textwidth}{0.3pt}]
    \titlespacing{\chapter}{0pt}{0pt}{\parskip}
\fi

% TODO: raggedright?
\ifnum\thesisChapterStyle=2
    \titleformat{\chapter}[display]
    {\Huge\bfseries\filcenter}
    {{\fontsize{50pt}{1em}\vspace{-4.2ex}\selectfont \textnormal{\thechapter}}}{1ex}{}[]
\fi

% Handle number of blank pages at chapter break
\if\thesisLayout 1
    \renewcommand{\cleardoublepage}{\clearpage}
\fi

% Name of chapters
% \addto\captionsenglish{\renewcommand{\abstractname}{}}
\addto{\captionsenglish}{\renewcommand{\contentsname}{Table of contents}}
\addto{\captionsenglish}{\renewcommand{\listfigurename}{List of figures}}
\addto{\captionsenglish}{\renewcommand{\listtablename}{List of tables}}
% \addto\captionsenglish{\renewcommand{\appendixname}{}}

% \bibname seems to be reset at \begin{document} (workaround in main)
\newcommand{\thesisBibName}{References}
\addto{\captionsenglish}{\renewcommand{\bibname}{\thesisBibName}}

%%%%%%%%%%%%%%%%%%%%%%%%%%%%%%%%%%%%%%%%%%%%%%%%%%%%%%%%%%%%%%%%%%%%%%%%%%%%%%%
% COVER PAGE BACKGROUND
\usepackage{eso-pic}

\newcommand{\backgroundpic}[3]{
    \put(#1,#2){
    \parbox[b][\paperheight]{\paperwidth}{
    \centering
    \includegraphics[width=\paperwidth,height=\paperheight,keepaspectratio]{#3}}}}

% COLOUR FOR HEADERS
\definecolor{headerBrown}{RGB}{144,102,78}

\if\thesisType B
    \definecolor{thesisHeaderColor}{RGB}{126,180,56} % Green
\fi

\if\thesisType M
    \definecolor{thesisHeaderColor}{cmyk}{0.14,0,0,0.65} % Gray
\fi

%%%%%%%%%%%%%%%%%%%%%%%%%%%%%%%%%%%%%%%%%%%%%%%%%%%%%%%%%%%%%%%%%%%%%%%%%%%%%%%
% PATCHES AND FIXES

% Give error on include with missing file
\makeatletter
\patchcmd\@include\@input@\input{}{}
\makeatother

%%%%%%%%%%%%%%%%%%%%%%%%%%%%%%%%%%%%%%%%%%%%%%%%%%%%%%%%%%%%%%%%%%%%%%%%%%%%%%%
% OTHER
\usepackage{csquotes}               % Quotations, using the command
\usepackage{lipsum}                 % Generating Lorem Ipsum
\usepackage{enumitem}               % Provides control over lists

\usepackage[yyyymmdd]{datetime}     % Dates
\renewcommand{\dateseparator}{-}    % Modify date separator

\usepackage[breakable]{tcolorbox}
% To-do notes
\if\thesisStatus f
    \usepackage[disable]{todonotes}
\else
    \if\thesisStatus d
		\usepackage[colorinlistoftodos]{todonotes}
    \fi
\fi
\setlength{\marginparwidth}{2.5cm}

% Chemistry:
% \usepackage{mhchem}               % Isotopes
\usepackage{chemfig}                % Chemical structures

% NOTE: this is not the correct placement for these packages:
% \usepackage[export]{adjustbox}    % Alt. way of inserting subfigs
% \usepackage{multicol}             % Multiple columns

% \usepackage{marvosym}             % Symbols, e.g. euro, zodiac etc.
% \usepackage{helvet}               % Enables different fonts
% \usepackage{wrapfig}              % Wrap figures
% \usepackage{arydshln}             % Dashed \hline and other

% \usepackage{pdflscape}            % Landscape-mode
% \usepackage{verbatim}             % e.q. comment environment
% \usepackage{moreverb}             % List settings
% \usepackage{comment}              % comment environment

%%%%%%%%%%%%%%%%%%%%%%%%%%%%%%%%%%%%%%%%%%%%%%%%%%%%%%%%%%%%%%%%%%%%%%%%%%%%%%%
% REFERENCES
% hyperref should, in general, be loaded last to avoid problems.
% Hence, other packages should, most likely, be placed above this.

\if\thesisStatus f
    \newcommand{\thesisColorlinks}{false}
\else
    % change to false for black links in draft
    \newcommand{\thesisColorlinks}{true}
\fi

% Clickable links in final pdf
\usepackage[
    hidelinks,
    linktoc     = all,
    colorlinks  = \thesisColorlinks,
    filecolor   = blue,
    linkcolor   = blue,
    urlcolor    = blue,
    citecolor   = blue,
    anchorcolor = blue,
]{hyperref}
\hypersetup{pdfinfo = {
    Author   = {\thesisAuthor},
    Title    = {\thesisImprintTitle},
    Keywords = {\thesisKeywords}
}}

\usepackage{bookmark}                                   % Improve bookmarks
\usepackage{url}                                        % Clickable url links
\usepackage[acronym, toc]{glossaries}
\usepackage[noabbrev, nameinlink]{cleveref}                         % Clever references
% Optional: [capitalise, nameinlink] for "Table 1" with "Table" part of the
%   hyperlink instead of "table 1" with hyperlink "1".
% Note: use \cpageref to reference pages with correct hyperlinks

% CREF
%\crefname{equation}{}{}  % "(1.1)" instead of "equation (1.1)"

% NUMBERING:
\numberwithin{equation}{chapter}    % Number equations within chapter
\numberwithin{figure}{chapter}      % Number figures within chapter
\numberwithin{table}{chapter}       % Number tables within chapter
\counterwithout{footnote}{chapter}  % Number footnotes without chapter

% Temporary (as in for this document only) commands
\newcommand{\fobj}{f_\text{obj}}


%special letters
\newcommand{\R}{\mathbb{R}}
\newcommand{\C}{\mathbb{C}}
\newcommand{\Z}{\mathbb{Z}}
\newcommand{\N}{\mathbb{N}}
\newcommand{\F}{\mathcal{F}}
\newcommand{\PP}{\mathbb{P}}
\newcommand{\EE}{\mathbb{E}}

%functions
\renewcommand{\Re}{\textnormal{Re}}
\renewcommand{\Im}{\textnormal{Im}}
\DeclareMathOperator{\Var}{Var} % Variance
\DeclareMathOperator{\Cov}{Cov} % Covariance
\DeclareMathOperator{\Res}{Res} % Residue
\DeclareMathOperator{\Ker}{Ker} % Kernel
\DeclareMathOperator{\tr}{tr} 	% Trace
\newcommand{\scalprod}[2]{\left\langle#1, #2\right\rangle}

\DeclarePairedDelimiter\abs{\lvert}{\rvert}%
\DeclarePairedDelimiter\norm{\lVert}{\rVert}%

% Swap the definition of \abs* and \norm*, so that \abs
% and \norm resizes the size of the brackets, and the
% starred versions do not.
\makeatletter
\let\oldabs\abs
\def\abs{\@ifstar{\oldabs}{\oldabs*}}
%
\let\oldnorm\norm
\def\norm{\@ifstar{\oldnorm}{\oldnorm*}}
\makeatother

%symbols
% for := as
\newcommand*{\defeq}{%
    \mathrel{%
        \vcenter{%
            \baselineskip0.5ex \lineskiplimit0pt \hbox{\scriptsize.}\hbox{\scriptsize.}%
        }%
    }%
    =%
}

% vectors are boldface
\renewcommand{\vec}{\bm}


%settings

%spacing
\renewcommand{\arraystretch}{1.2}
\setlength{\arraycolsep}{3pt}

%units
\sisetup{
   output-decimal-marker = {,},
   exponent-product = \ensuremath{\cdot}
}

%text
\makeatletter
   \newlength{\textSize}
   \setlength{\textSize}{\f@size pt}
\makeatother
\setlength{\parindent}{0pt}             % No indentation on new paragraph
\setlength{\parskip}{\textSize}         % Blankline on new paragraph
\setstretch{1.15}

\newcommand{\todowrt}[2][]{\todo[color=green,inline,#1]{#2}}
\newcommand{\tododec}[2][]{\todo[color=yellow,inline,#1]{#2}}
\newcommand{\todoblk}[2][]{\todo[color=red,inline, #1]{#2}}
\newcommand{\todocit}[2][]{\todo[color=blue!30!white, #1]{#2}}

%paper and text body
\geometry{
   a4paper,
   centering,
}


%lettrine settings
\setcounter{DefaultLines}{3}


\newacronym{pml}{PML}{Perfectly Matched Layer}

%bibliography
\addbibresource{sources.bib}

\title{Inverse design of quantum acoustic devices}
\author{David Hambr\ae{}us}
\date{\textit{\today}}

\begin{document}

\maketitle

\todo[inline]{Todonotes are organized as follows:}
\todo[inline]{General comment / question}
\todowrt{Things that could be done now, no further
simulations/consultation needed}
\tododec{Things that could be done now but I am not sure if I should, or
how to do it}
\todoblk{Things that can't be done yet because they depend on other
things, e.g.\ results}
\todocit[inline]{Citation needed}

\listoftodos[List of Todos]

\tableofcontents

\printglossaries{}

\newpage

\section{Introduction}

\tododec{Introduction to quantum acoustics\ldots I need to read more
literature I think}
\tododec{Restructure a little... I would like to talk about inverse design first
and quantum acoustics second. Talk about how inverse design is a concept that
has been applied to nanophotonics but not to quantum acoustics yet. Then talk
about why we care about quantum acoustics. It feels a little bit forced to do it
in that order though, talking about quantum acoustics first might be a better
idea, since that would naturally lead one to introduce the problem of design.}

Conventionally, when designing these components, the designer comes up with a
design through intuition and parametrizes it with a couple of parameters.
For example they may believe that a structure with periodically placed circular
holes should yield a device that performs the desired function.
% TODO: She? I wrote he first but that will only reinforce the stereotype that
% the are no women in nano science.
The parameters that are unknown might then be the distance between neighbouring
holes and the radius of the holes.
To find the optimal device they would then systematically test parameter values
to see which give the best performance in a simulation of the device.
This brute force method of design limits the possible number of parameters to a
very small number.
If there are 10 different values to test for each parameter, the even 6
parameters would require 1,000,000 simulations.
One can of course use smarter optimization algorithms like bayesian optimization
\todocit{cite something, check Ida's thesis maybe}
or particle swarm optimization
\todocit{cite the thing the danish guys cited}
to decrease the number of simulations needed, but it will still be of the same
order.

A different approach that has been gaining some popularity is
\emph{inverse design with adjoint simulation}.
\todocit{maybe cite \emph{inverse design in nanophotonics}}
The idea is that if the gradient of the figure of merit
with respect to the parameters can be calculated, then we can use gradient based
optimization methods, which converge much faster, even if the number of
parameters is very large. With these methods, one hopes to be able to search
among a much more general class of designs for the optimal one.
% Su? Isn't it Jelena?
\citeauthor{spins2019} has developed software that successfully uses inverse design for
nanophotonic devices \cite{spins2019}.

With this thesis, we explore the possibility of extending this paradigm to
acoustic devices. In order to do so, we attempt to design a phononic beam
splitter.

\subsection{Thesis outline}

\section{Theory}

\subsection{Acoustic waves and waveguides}

In order to efficiently model the deformation and stresses in a solid material,
a linear elasticity model is often assumed.
For small deformations, solid materials obey Hooke's law which in it's full form
looks like
\[
	\sigma_{ij} = C_{ijkl} \epsilon_{jl}
\]
where $\sigma$ is the stress tensor, $C$ the elasticity tensor which is a
property of the material, and 
$\epsilon \defeq \nabla \vec u + (\nabla \vec u)^T$
is the strain tensor.
This equation is linear in $\vec u$, hence the name \emph{linear} elasticity.
Using this and newtons equations of motion, the equation governing the dynamics
is obtained:
\[
	\rho \ddot{\vec{u}} = \nabla \cdot \sigma + \vec F.
\]
where $\rho$ is the density, $\vec{u}$ is the displacement and $\vec F$ is the
externally applied force.
\tododec{Proper derivation for equation below? It is relatively
straightforward but requires some effort}
Assuming a time harmonic solution
$\vec u(\vec x, t) = \vec u(x) e^{i \omega t}$ and plugging this into
the governing equations yields
\begin{equation}\label{eq:gov_eq}
	-\rho \omega^2 u_i = \partial_j C_{ijkl} \partial_k u_l + F_i
\end{equation}
written in index notation for clarity. 

\todowrt{
With no external forces we get traveling modes=eigenvalues.
Explain how and why periodicity means only certain modes can propagate.
Good resource in Chan's thesis, or maybe solid state physics book?
% Bloch state / Bloch's theorem
}

\tododec{At some point write about phonons? I haven't really had to care about
phonons so if I talk about it it's just for applications...}

\todowrt{Show our mode as example of this, and include band diagram}

\todowrt{Write about PML design and why we need it: Simulating infinite
	waveguides isn't possible because of the finite computing power. Also we
don't care about stuff far away, just that there are no reflections that can
interfere.}

\subsection{Inverse Design}

Inverse design is a design paradigm where the design of a device is guided fully by
the desired characteristics.
These desired characteristics are quantified through what is called an objective
function%
\footnote{
	Also called \emph{figure of merit (FoM)} by some.%
}%
, which I will denote $\fobj$,
that should be maximized.
When coupled with \emph{adjoint simulation}, which is a clever way to compute
gradients, and gradient based optimization
algorithms, this is a very powerful methodology.

An overview of the design process is as follows:
\begin{enumerate}
	\item Initialize a random device design.
	\item\label{it:grad} Calculate the gradient of the design through the adjoint method.
	\item Update the device design using the gradient according to the optimization algorithm.
	\item If the device performance is good enough, terminate optimization, else
		return to step~\ref{it:grad}.
\end{enumerate}

\subsubsection{Adjoint Simulation}

Adjoint simulation is a way to compute the gradient of $\fobj$ with respect to
the design, which in our case means with respect to the material parameters.
I will in this section first give a general derivation, followed by the case of
inverse design in acoustics.

\paragraph{General Derivation}

Let $\fobj$ be a function which depends on some (large) vector $v$.
The vector $v$ can be calculated by solving the linear equation
$A v = b$, where $b$ is a fixed vector and $A$ is a matrix that depends on a
vector of design parameters $p$.
The overall goal is to find the parameters $p$ that maximize the objective
function $\fobj$.
The goal of adjoint simulation is to find $\diff{\fobj}{p}$.
This can be expanded through the chain rule as
\[
	\diff{\fobj}{p} = \diff{\fobj}{v} \diff{v}{p}.
\]
To find the latter factor we do
\begin{align*}
	\diff{}{p} [Av = b] &\implies \diff{A}{p} v + A \diff{v}{p} = 0\\
						&\implies \diff{v}{p} = -A^{-1} \diff{A}{p} v
\end{align*}
which gives
\begin{align}
	\diff{\fobj}{p} &= - \diff{\fobj}{v} A^{-1} \diff{A}{p}
	v\label{eq:no_adj_sim}\\
	&= - \left(A^{-T} \diff{\fobj}{v}^T \right)^T \diff{A}{p} v
\end{align}
The first factor of this product is the solution to the \emph{adjoint problem}
\begin{equation}
	A^T \tilde v = \diff{\fobj}{v}^T,
\end{equation}
hence the name adjoint simulation.
As it turns out, $A$ is often symmetric which means that this is simply a normal
simulation but with $\diff{\fobj}{v}^T$ as the source.
Thus, to obtain the derivative we just need to run an additional
simulation with a different input.

Now you might be wondering: what have we gained by this?
Let $n$ be the dimension of $v$, $m$ the dimension of $p$ and $l$ the dimension
of $b$.
This means that $A$ is a matrix with dimension $l\times n$ and $\diff{A}{p}$ is
a three-tensor with dimension $m\times l\times n$.
Thus calculating $A^{-1} \diff{A}{p}$ directly involves solving $Ax = w$ for a
three-tensor, and calculating $A^{-1} \left(\diff{A}{p} v\right)$
involves solving for a matrix, both of which are orders of magnitude more
computationally expensive than solving for a vector.

\paragraph{Specific derivation with acoustics}

\tododec{Derive specific case (with functional analysis):
	borrowing heavily from the mathy notes is possible.
	Derivation is theoretically possible without specifying that the objective function
	is an overlap integral, only
	using equation~\eqref{eq:gov_eq},
	but the equations would be \emph{a lot} longer and more abstruse,
	so I don't think that it is a good idea.
}

\subsubsection{Optimization Algorithms}

\tododec{General paragraph on the benefits of gradient based
optimization algorithms vs other algorithms? And general overview of the
optimization process: simulate -- compute gradient -- step}

\todowrt{Paragraph describing regular gradient descent}

\todowrt{Paragraph describing the ADAM algorithm}

\section{Methods}

The aim of this thesis is to use inverse design to find a phononic beamsplitter,
a task that can be divided into three parts: 
First, some definitions of what
should be designed and what constitutes a ``good'' design needs to be made.
Second, we need a way to calculate the gradient of the ``goodness'' with respect
to the design.
And lastly, we need a gradient based optimization algorithm to find the optimal
design.
All of this will be described in this chapter.

\subsection{Design}

The device design to be optimized can be seen in figure~\ref{fig:bs-design}

\begin{figure}[htpb]
	\centering
	\def \a{0.5}
\def \w{1.0}
\def \hx{0.13}
\def \hy{0.3}

\tikzset{
	unitcell/.pic={
		\draw[pic actions] (-0.5*\a, -0.5*\w) rectangle (0.5*\a, 0.5*\w);
		\draw[fill=white] (0, 0) circle [x radius=\hx, y radius=\hy];
	}
}

\begin{tikzpicture}[scale=0.7]
	\def \designx{4.0}
	\def \designy{4.0}
	\def \outputh{1.0}
	\def \nunitcells{16}
	\def \nnonpmls{10}

	% Coordinate system
	\draw[gray, thick, ->] (0,0) -- (0,-1) node [anchor=west] {$x$};
	\draw[gray, thick, ->] (0,0) -- (1,0) node [anchor=west] {$y$};

	% Input waveguide
	\path
		(-0.5*\a, 0) pic[transform shape] {unitcell}
		(-1.5*\a, 0) pic[transform shape] {unitcell}
		(-2.5*\a, 0) pic[transform shape] {unitcell}
		(-4.0*\a, 0) node {$\cdots$}
		(-5.5*\a, 0) pic[transform shape] {unitcell}
		(-6.5*\a, 0) pic[transform shape] {unitcell}
		(-7.5*\a, 0) pic[transform shape] {unitcell};
	\draw[red, ultra thick]
		(-7*\a, -0.5*\w) --
		(-7*\a, +0.5*\w);
	\begin{scope}[dash=on 1pt off 1pt phase 0pt]
	\path
		(-8.5*\a, 0) pic[transform shape] {unitcell}
		(-9.5*\a, 0) pic[transform shape] {unitcell}
		(-10.5*\a, 0) pic[transform shape] {unitcell}
		(-12.0*\a, 0) node {$\cdots$}
		(-13.5*\a, 0) pic[transform shape] {unitcell};
	\end{scope}

	% Design area
	\draw (0, -\designx / 2) rectangle (\designy, \designx / 2);
	\node at (\designy / 2, 1) {Design Area};
	\node[left] at (\designy, 0) {$d_x$};
	\node[above] at (\designy/2, -\designx/2) {$d_y$};

	% Output waveguide
	\begin{scope}[xshift=\designy cm, yshift=\outputh cm]
	\path
		(0.5*\a, 0) pic[transform shape] {unitcell}
		(1.5*\a, 0) pic[transform shape] {unitcell}
		(2.5*\a, 0) pic[transform shape] {unitcell}
		(4.0*\a, 0) node {$\cdots$}
		(5.5*\a, 0) pic[transform shape] {unitcell}
		(6.5*\a, 0) pic[transform shape] {unitcell}
		(7.5*\a, 0) pic[transform shape, fill=blue] {unitcell};
	\begin{scope}[dash=on 1pt off 1pt phase 0pt]
	\path
		(8.5*\a, 0) pic[transform shape] {unitcell}
		(9.5*\a, 0) pic[transform shape] {unitcell}
		(10.5*\a, 0) pic[transform shape] {unitcell}
		(12.0*\a, 0) node {$\cdots$}
		(13.5*\a, 0) pic[transform shape] {unitcell};
	\end{scope}
	\end{scope}
	\begin{scope}[xshift=\designy cm, yshift=-\outputh cm]
	\path
		(0.5*\a, 0) pic[transform shape] {unitcell}
		(1.5*\a, 0) pic[transform shape] {unitcell}
		(2.5*\a, 0) pic[transform shape] {unitcell}
		(4.0*\a, 0) node {$\cdots$}
		(5.5*\a, 0) pic[transform shape] {unitcell}
		(6.5*\a, 0) pic[transform shape] {unitcell}
		(7.5*\a, 0) pic[transform shape, fill=blue] {unitcell};
	\begin{scope}[dash=on 1pt off 1pt phase 0pt]
	\path
		(8.5*\a, 0) pic[transform shape] {unitcell}
		(9.5*\a, 0) pic[transform shape] {unitcell}
		(10.5*\a, 0) pic[transform shape] {unitcell}
		(12.0*\a, 0) node {$\cdots$}
		(13.5*\a, 0) pic[transform shape] {unitcell};
	\end{scope}
	\end{scope}
	\draw[|-|]
		(15*\a + \designy, -\outputh) -- node[right] {$s$}
		(15*\a + \designy, \outputh);
		
\end{tikzpicture}

	\caption{
		Device design to be optimized.
		At the red line, a wave traveling right is excited.
		On the blue unit cells is where the output is measured.
		The dashed unit cells are \gls{pml}
	}
	\label{fig:bs-design}
\end{figure}

\tododec{Write about why we use the mode we use, and why I clamp the bottom. Can
reference Johan and Pauls paper}

\subsubsection{Level-set}

\todowrt{Description of what level-set is: A way of storing a boundary between
two regions; and how it works: Signed distance function}

\todowrt{Description of it's advantages: Easily evolved (level-set equation), no
connectivity issues, see level-set book}

\tododec{Write about how I use it? This feels like it should come after I've
talked about computing the derivative.}

\subsubsection{Objective function}

\tododec{Maybe I'll only mention that $f$ is an integral of the displacement
	field in the theory and here give the specific formula:
}
\begin{equation}
	\fobj = \int_{\Omega_1} \vec{m_1}^*(\vec{x}) \vec{u}(\vec{x})
	\dl{x} + \int_{\Omega_2} \vec{m_2}^*(\vec{x}) \vec{u}(\vec{x})
\end{equation}

\tododec{Paragraph about pure part of objective function, enforcing the minimum
feature size}

\subsection{Adjoint Simulation}

\tododec{Give the explicit formula for the gradient now that the objective
function has been fully defined}

\subsection{Optimization}

\tododec{Describe what optimization algorithm was used, as well as how this
	changed during the simulation. E.g.\ first 200 iterations ADAM; next ADAM
	but with sigmoid function application; sigmoid + feature size; and finally level-set.
}

\section{Results}

\section{Discussion}

\section{Conclusions}

\printbibliography{}

\end{document}
