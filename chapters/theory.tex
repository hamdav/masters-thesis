\chapter{Theory}

\section{Acoustic waves and waveguides}

In order to efficiently model the deformation and stresses in a solid material,
a linear elasticity model is often assumed.
For small deformations, solid materials obey Hooke's law which in it's full form
looks like
\[
	\sigma_{ij} = C_{ijkl} \epsilon_{jl}
\]
where $\sigma$ is the stress tensor, $C$ the elasticity tensor which is a
property of the material, and 
$\epsilon \defeq \nabla \vec u + (\nabla \vec u)^T$
is the strain tensor.
This equation is linear in $\vec u$, hence the name \emph{linear} elasticity.
Using this and newtons equations of motion, the equation governing the dynamics
is obtained:
\[
	\rho \ddot{\vec{u}} = \nabla \cdot \sigma + \vec F.
\]
where $\rho$ is the density, $\vec{u}$ is the displacement and $\vec F$ is the
externally applied force.
\tododec{Proper derivation for equation below? It is relatively
straightforward but requires some effort}
Assuming a time harmonic solution
$\vec u(\vec x, t) = \vec u(x) e^{i \omega t}$ and plugging this into
the governing equations yields
\begin{equation}\label{eq:gov_eq}
	-\rho \omega^2 u_i = \partial_j C_{ijkl} \partial_k u_l + F_i
\end{equation}
written in index notation for clarity. 

\todowrt{
With no external forces we get traveling modes=eigenvalues.
Explain how and why periodicity means only certain modes can propagate.
Good resource in Chan's thesis, or maybe solid state physics book?
% Bloch state / Bloch's theorem
}

\tododec{At some point write about phonons? I haven't really had to care about
phonons so if I talk about it it's just for applications...}

\todowrt{Show our mode as example of this, and include band diagram}

\todowrt{Write about PML design and why we need it: Simulating infinite
	waveguides isn't possible because of the finite computing power. Also we
don't care about stuff far away, just that there are no reflections that can
interfere.}

\section{Inverse Design}

Inverse design is a design paradigm where the design of a device is guided fully by
the desired characteristics.
These desired characteristics are quantified through what is called an objective
function%
\footnote{
	Also called \emph{figure of merit (FoM)} by some.%
}%
, which I will denote $\fobj$,
that should be maximized.
When coupled with \emph{adjoint simulation}, which is a clever way to compute
gradients, and gradient based optimization
algorithms, this is a very powerful methodology.

An overview of the design process is as follows:
\begin{enumerate}
	\item Initialize a random device design.
	\item\label{it:grad} Calculate the gradient of the design through the adjoint method.
	\item Update the device design using the gradient according to the optimization algorithm.
	\item If the device performance is good enough, terminate optimization, else
		return to step~\ref{it:grad}.
\end{enumerate}

\subsection{Adjoint Simulation}

Adjoint simulation is a way to compute the gradient of $\fobj$ with respect to
the design, which in our case means with respect to the material parameters.
I will in this section first give a general derivation, followed by the case of
inverse design in acoustics.

\subsubsection{General Derivation}

Let $\fobj$ be a function which depends on some (large) vector $v$.
The vector $v$ can be calculated by solving the linear equation
$A v = b$, where $b$ is a fixed vector and $A$ is a matrix that depends on a
vector of design parameters $p$.
The overall goal is to find the parameters $p$ that maximize the objective
function $\fobj$.
The goal of adjoint simulation is to find $\diff{\fobj}{p}$.
This can be expanded through the chain rule as
\[
	\diff{\fobj}{p} = \diff{\fobj}{v} \diff{v}{p}.
\]
To find the latter factor we do
\begin{align*}
	\diff{}{p} [Av = b] &\implies \diff{A}{p} v + A \diff{v}{p} = 0\\
						&\implies \diff{v}{p} = -A^{-1} \diff{A}{p} v
\end{align*}
which gives
\begin{align}
	\diff{\fobj}{p} &= - \diff{\fobj}{v} A^{-1} \diff{A}{p}
	v\label{eq:no_adj_sim}\\
	&= - \left(A^{-T} \diff{\fobj}{v}^T \right)^T \diff{A}{p} v
\end{align}
The first factor of this product is the solution to the \emph{adjoint problem}
\begin{equation}
	A^T \tilde v = \diff{\fobj}{v}^T,
\end{equation}
hence the name adjoint simulation.
As it turns out, $A$ is often symmetric which means that this is simply a normal
simulation but with $\diff{\fobj}{v}^T$ as the source.
Thus, to obtain the derivative we just need to run an additional
simulation with a different input.

Now you might be wondering: what have we gained by this?
Let $n$ be the dimension of $v$, $m$ the dimension of $p$ and $l$ the dimension
of $b$.
This means that $A$ is a matrix with dimension $l\times n$ and $\diff{A}{p}$ is
a three-tensor with dimension $m\times l\times n$.
Thus calculating $A^{-1} \diff{A}{p}$ directly involves solving $Ax = w$ for a
three-tensor, and calculating $A^{-1} \left(\diff{A}{p} v\right)$
involves solving for a matrix, both of which are orders of magnitude more
computationally expensive than solving for a vector.

\subsubsection{Specific derivation with acoustics}

Now we turn to the specific case of acoustic devices.
Here $A v = b$ is replaced by the dynamic equation \todo{dynamic equation?
Better name} of linear elasticity:
\begin{equation}
	-\left(\rho \omega \delta_{il} + \partial_j C_{ijkl} \partial_k\right) u_l =
	F_i
\end{equation}

\tododec{Derive specific case (with functional analysis):
	borrowing heavily from the mathy notes is possible.
	Derivation is theoretically possible without specifying that the objective function
	is an overlap integral, only
	using equation~\eqref{eq:gov_eq},
	but the equations would be \emph{a lot} longer and more abstruse,
	so I don't think that it is a good idea.
}

\subsection{Optimization Algorithms}

\tododec{General paragraph on the benefits of gradient based
optimization algorithms vs other algorithms? And general overview of the
optimization process: simulate -- compute gradient -- step}

\todowrt{Paragraph describing regular gradient descent}

\todowrt{Paragraph describing the ADAM algorithm}
