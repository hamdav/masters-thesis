\chapter{Results}

The continuous optimization yielded devices with near perfect performance.
However, the optimization never converged.
After reaching a high value, the algorithm invariably descended to very low
levels.
This failure to converge prevented a gradual introduction of penalty terms for
in-between values, which would have allowed for a smooth transition to the
level-set design paradigm.
Nevertheless, the level-set methods were also tried,
though since the best devices of the continuous optimization were very far from
binary, manual designs were used as a starting point.
The level-set optimization fared worse in some regards, better in others.
The algorithm did converge after a couple hundred iterations.
However, the final value to which it converged was only 25\% of a perfect
performance.

\todoblk{%
	Paragraph summarizing the results: best figure of merit, no of iterations
	and maybe time for simulations.
}

\todoblk{%
	More specifics... what should I even put here, and how should I structure it?
}
\todoblk{%
	A word on errors and why they happen. E.g. the density -> 0 numerically
	unstable thing
}

\section{Continuous optimization}

\begin{figure}[htpb]
	\centering
	%\includegraphics{}
	\missingfigure{Optimal design}
	\caption{}
	\label{fig:optimal_cont_design}
\end{figure}

\begin{figure}[htpb]
	\centering
	%\includegraphics{}
	\missingfigure{Convergence plot}
	\caption{}
	\label{fig:convergence_plot_cont}
\end{figure}

\section{Level-set optimization}

\begin{figure}[htpb]
	\centering
	%\includegraphics{}
	\missingfigure{Optimal design}
	\caption{}
	\label{fig:optimal_bin_design}
\end{figure}

\begin{figure}[htpb]
	\centering
	%\includegraphics{}
	\missingfigure{Convergence plot}
	\caption{}
	\label{fig:convergence_plot_bin}
\end{figure}
