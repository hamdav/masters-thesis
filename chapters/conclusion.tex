\chapter{Concluding remarks}

Phononics have the potential to become a vital part of quantum information
processing architecture, but is yet plagued by being difficult to design.
This study set out to investigate whether inverse design with adjoint simulation
could be used to design better devices.
I first examined theoretically the applicability of adjoint simulation on
phononics, from which I concluded that the methods should work in theory.

The second aim of this thesis was to demonstrate the utility of this method by
designing a phononic beamsplitter.
To do that, the process was split into two stages, a first where the material
was allowed to vary continuously, and a second where a binary design was
enforced using level-set methods.
The continuous optimization yielded near perfect performance, validating the
theoretical derivations.
These results indicate that the inverse design concept can be very useful for
designing phononic devices going forward.
The level-set optimization did not yield as clear and good results, though
decent performance was still achieved.
The author remains hopeful that the problems with the latter stage can be solved
though, which would be a major step forward.

\section{Future research}

There are a great number of potential paths that can be explored from this
point, and the positive results presented here warrant further efforts in this
area.
One important improvement that should be investigated is the sensitivity of the
device to small changes in the design.
If it is very sensitive, imperfections in fabrication could be detrimental to
the devices performance.
There may be some ways of mitigating this however, for example by running
multiple simulations with small perturbations in the design and averaging them
to obtain the objective function.
Another potential path of exploration is the limiting of the feature size.
This could also be done by augmenting the objective function, but this time with
a pure part that punishes small features.

In addition to method improvements, another path is to apply this to designing
other types of devices. For example, waveguide bends seem to be well suited for
this type of design, and would be useful if one wishes to use phononic
waveguides for routing excitations around on a chip. It may also be possible to
inverse design hybrid devices that use both optics and mechanics for example,
though that would likely require a significant effort.
