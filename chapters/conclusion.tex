\chapter{Concluding Remarks}

Phononics has the potential to become a vital part of both quantum and
classical information processing architecture, but devices are still limited to
those that can be investigated through analytical means and/or be optimized with
brute force methods. This study set out to investigate whether inverse design
with adjoint simulation could be used to design so far unrealized devices and
enable little explored applications.

We first examined theoretically the applicability of adjoint simulation on
phononics, from which we concluded that the methods should work.
The second aim of this thesis was to demonstrate the utility of this method by
designing a phononic beamsplitter as a proof-of-concept.
To do that, the process was split into two stages, a first where the material
was allowed to vary continuously, to obtain an approximate design as the
starting point for the second stage, where a binary design was
enforced using level-set methods.
The continuous optimization yielded near perfect performance, validating the
theoretical derivations.
The level-set optimization never did achieve perfect performance, but
good performance was still reached.
These results indicate that the inverse design concept can be useful for
designing traveling-wave phononic devices going forward.
We remain hopeful that the problems with the latter stage can be solved
though, which would be a major step forward.

\section{Future Research}

There are a great number of potential paths that can be explored from this
point, and the positive results presented here warrant further efforts in this
area.
One important improvement that should be investigated is the sensitivity of the
device to small changes in the design.
If it is very sensitive, imperfections in fabrication could be detrimental to
the devices performance.
There may be some ways of mitigating this however, for example by running
multiple simulations with small perturbations in the design and averaging them
to obtain the objective function.
Another potential path of exploration is the limiting of the feature size.
This could be done by augmenting the objective function, adding
a pure part that punishes small features.

After these issues have been solved, one could make the model more realistic,
for example by adding a substrate instead of the fixed bottom used in this
thesis.
Ultimately, the designs need to be fabricated and measured experimentally as
well to confirm that the designed devices are functional.

In addition to method improvements, another path is to apply this to designing
other types of devices. For example, waveguide bends seem to be well suited for
this type of design, and would be useful if one wishes to use phononic
waveguides for routing excitations around on a chip. It may also be possible to
inverse design hybrid devices that use both optics and mechanics for example,
though that would likely require a significant effort.
