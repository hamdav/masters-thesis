\chapter{Introduction}
\todoblk{After thesis is done, check that I've used cref and not ref}
\todoblk{References before the dot or after? And space or no space?}
\todoblk{ctrl+F for all the times when I write I / we and choose one of them...}
\tododec{Write about why I am using \emph{silicon}}

\tododec{Introduction to quantum acoustics\ldots I need to read more
literature I think}
\tododec{Restructure a little... I would like to talk about inverse design first
and quantum acoustics second. Talk about how inverse design is a concept that
has been applied to nanophotonics but not to quantum acoustics yet. Then talk
about why we care about quantum acoustics. It feels a little bit forced to do it
in that order though, talking about quantum acoustics first might be a better
idea, since that would naturally lead one to introduce the problem of design.}

Conventionally, when designing these components, the designer comes up with a
design through intuition and parametrizes it with a couple of parameters.
For example they may believe that a structure with periodically placed circular
holes should yield a device that performs the desired function.
% TODO: She? I wrote he first but that will only reinforce the stereotype that
% the are no women in nano science.
The parameters that are unknown might then be the distance between neighbouring
holes and the radius of the holes.
To find the optimal device they would then systematically test parameter values
to see which give the best performance in a simulation of the device.
This brute force method of design limits the possible number of parameters to a
very small number.
If there are 10 different values to test for each parameter, even six
parameters would require 1,000,000 simulations.
One can of course use smarter optimization algorithms like bayesian optimization
\todocit{cite something, check Ida's thesis maybe}
or particle swarm optimization\cite{zhang_compact_2013}
to decrease the number of simulations needed, but it will still be of the same
order.

A different approach that has been gaining some popularity is
\emph{inverse design with adjoint simulation}.%
\cite{molesky_inverse_2018}
The idea is that if the gradient of the figure of merit
with respect to the parameters can be calculated, then we can use gradient based
optimization methods, which converge much faster, even if the number of
parameters is very large. With these methods, one hopes to be able to search
among a much more general class of designs for the optimal one.
% Su? Isn't it Jelena?
\citeauthor{spins2019} has developed software that successfully uses inverse design for
nanophotonic devices \cite{spins2019}.

With this thesis, we explore the possibility of extending this paradigm to
acoustic devices. In order to do so, we attempt to design a phononic beam
splitter.

\todowrt{Say something about the fact that I derived the acoustic case myself}

\section{Thesis outline}
