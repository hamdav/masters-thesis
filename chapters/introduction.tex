\chapter{Introduction}

In recent years, the research into quantum devices of different kinds has
significantly intensified.
Though many people are focusing on superconducting circuits, where quanta of
microwave-frequency light are manipulated, there has also been
a growing interest in a different medium for quantum information: sound.
More precisely, acoustic waves in solid materials.
Just like how light, or electromagnetic waves, come in quantized packets of energy called
photons,
so too does acoustic waves and those packets are called phonons.
Just recently, in 2019, researchers coupled an acoustic resonator to a transmon
qubit and were able to directly resolve the number of phonons~\cite{arrangoiz-arriola_resolving_2019}.
Acoustic or mechanical wave devices, also called phononic devices, have already
found a multitude of applications in radio-signal processing and sensing, and
research into new and better devices for these applications continues in
parallel with devices for quantum applications~\cite{laer_controlling_2019}.

Possible quantum applications of phononic devices are many.
One of them is quantum memory~\cite{sete_high-efficiency_2015}.
Regular computers have both memory and a processing unit that
retrieves data from memory, applies operations on it, and then returns it to
memory.
Keeping the data inside the processing unit would be inefficient,
since the processing unit would then have to be very large.
The same goes for quantum computers: storing the quantum information in
the same place where the computation happens is not scalable.
Instead, a quantum memory could be used to store information while it is not
directly needed for the computation.
The reason for using mechanical resonators as memory is that long lifetimes, on
the order of seconds at \qty{5}{\giga\hertz}, have been achieved in such
resonators.
Additionally, acoustic waves have a much smaller wavelength compared to
electromagnetic waves at the same frequency, meaning that devices can be more
compact~\cite{maccabe_phononic_2020}.
Another application of quantum acoustics worth mentioning is
coherent transduction between microwave and optical photons.
This would enable communication between physically separated superconducting
circuit based quantum computers~\cite{laer_controlling_2019}.
A third possible application is doing quantum information processing purely in
mechanical elements~\cite{qiao2023developing}.
\todocit{raphael doesn't like this ref}

One important problem with such quantum acoustic devices is that they can be hard to design.
Currently they are often designed by hand,
through analytical derivations or analytically motivated guesswork,
combined with brute force parameter sweeps.
However, this severely limits the designs that can be investigated:
sometimes analytical derivations are impossible, and brute force parameter
sweeps quickly become computationally intractable.
A parameter sweep of just 6 parameters with 10 different values for each
requires \num{1000000} simulations.
One can of course use smarter optimization algorithms like Bayesian optimization
or particle swarm optimization~\cite{schneider2019benchmarking,zhang_compact_2013}
to decrease the number of simulations needed,
but they will still struggle with high dimensional
problems~\cite{chen_measuring_2015}.

A different approach that has been gaining some popularity in the field of
nano-photonics is \emph{inverse design with adjoint
simulation}~\cite{molesky_inverse_2018}.
The idea is that if the gradient of the figure of merit of the device
with respect to the design parameters can be calculated, then gradient based
optimization methods can be used.
These converge much faster, even if the number of
parameters is very large. Thus, one hopes to be able to search
among a much more general class of designs for the optimal one.
This has been successfully applied to photonic devices,
where a wide variety of components have been designed~\cite{spins2019}.
In some cases, for example the waveguide bend, the inverse-designed device could
be made much more compact than conventionally designed
bends~\cite{jensen_systematic_2004}.
In other cases, for example the vertically incident wavelength-demultiplexing
grating coupler, there are no other known conventional methods of designing the
device~\cite{piggott_inverse_2014}.
Before their application in nano-photonics, closely related methods were used in
structural mechanics under the name of topology
optimization, for example to optimize the stiffness with limited material~\cite{sigmund_topology_2013}.
It has also been used for engineering some properties (e.g.\ bandgap) of
both photonic and phononic crystals~\cite{yi_comprehensive_2016}.
However, to our knowledge, it has not yet been used for traveling-wave phononic devices.

Phononic devices have in general been studied less than photonic
devices, and the library of known devices is smaller.
With this thesis, we explore the possibility of extending the inverse-design paradigm to
traveling-wave phononic devices.
Since both acoustics and electromagnetics are wave phenomena, there are many
analogies to be drawn, but there are also important differences.
For example, the different boundary conditions mean that acoustics support surface waves
while electromagnetics does not.
We show in this work that the adjoint method is applicable to traveling-wave phononics,
and derive the form of the equations.
As a proof of concept, we attempt to design a phononic 50/50 beamsplitter.
The beamsplitter is conceptually one of the simplest devices imaginable,
and photonic beamsplitters have been studied in great detail for decades
and found a multitude of uses in different experiments.
However, there is still no standard implementation for phononic beamsplitters,
though~\cite{qiao2023developing} has demonstrated a working beamsplitter for
surface acoustic waves.
There are two major differences between their design and the one attempted here.
Firstly, they are using surface acoustic waves while we are using waves propagating
through patterned silicon waveguides.
Secondly, their beamsplitter reflects one of the arms of the beamsplitter back towards
the source, whereas we have one input waveguide and two separate outputs.

\section{Aim and Thesis Outline}

The aims of this thesis are:
\begin{itemize}
	\item Rederive the equations for inverse design with adjoint simulation in
		the case of phononics and confirm that the methods are theoretically
		applicable.
	\item Implement the methods and use them to design a phononic beamsplitter.
\end{itemize}

Chapter 2 presents some of the theory on solid mechanics and acoustic waves needed to
understand the thesis.
A band diagram over the modes of the waveguide used is also shown there.
In chapter 3, the inverse design process is presented.
First, the general adjoint method is showcased,
followed by a derivation in the specific case of acoustics.
Then, gradient based optimization algorithms are discussed and the ones used in
this thesis are presented.
Chapter 4 describes the specifics of the simulations performed in this work:
both a specification of the skeleton of the devices, how the input wave was excited,
and how the designs were parametrized. It also describes the specifics of the
optimization algorithm, including parameter values.
Chapter 5 then presents and analyzes the results of the simulations performed
and Chapter 6 summarizes the conclusions drawn from the results and describes
possible future research.
