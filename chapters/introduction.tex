\chapter{Introduction}
\todoblk{After thesis is done, check that I've used cref and not ref}
\todoblk{References before the dot or after? And space or no space?}
\todoblk{ctrl+F for all the times when I write I / we and choose one of them...}
\tododec{Write about why I am using \emph{silicon}}

\tododec{Introduction to quantum acoustics\ldots I need to read more
literature I think}
\tododec{Restructure a little... I would like to talk about inverse design first
and quantum acoustics second. Talk about how inverse design is a concept that
has been applied to nanophotonics but not to quantum acoustics yet. Then talk
about why we care about quantum acoustics. It feels a little bit forced to do it
in that order though, talking about quantum acoustics first might be a better
idea, since that would naturally lead one to introduce the problem of design.}

In recent years, the research into quantum devices of different kinds has
significantly intensified.
Much of it is in one way or another connected to the
construction or operation of a quantum computer.
Though many people are focusing on superconducting circuits, where quanta of
microwave-frequency photons are manipulated, there has also been
a growing interest in a different medium for quantum information: sound.
More precisely, acoustic waves in solid materials.
Just like how light, or electromagnetic waves, come in quantized packets of energy called
photons,
so too does acoustic waves and we call those packets phonons.
Just recently, in 2019, researchers coupled an acoustic resonator to a transmon
qubit and were able to directly measure the presence or absence of single
phonons.

Possible applications of quantum acoustic devices are many.
One of them is it's use in quantum memory.
Regular computers have both memory and a processing unit that
retrieves data from memory, applies operations on it, and then returns it to
memory.
Keeping all of the data at the same place where the computing happens would be
very inefficient.
The same goes for quantum computers: storing all of the quantum information in
the same place where the computation happens is probably not a scalable plan.
Another application of quantum acoustics worth mentioning is
coherent transduction between microwave and optical photons.
This would enable communication between physically separated superconducting
circuit based quantum computers.\cite{laer_controlling_2019}

One important problem with such devices is that they can be hard to design.
Currently they are often designed by hand,
through analytically motivated guesswork combined with brute force parameter
sweeps.
However, this severely limits the designs that can be investigated.
A parameter sweep of just 6 parameters with 10 different values for each
requires \num{1000000} simulations.
One can of course use smarter optimization algorithms like bayesian optimization
or particle swarm optimization\cite{schneider2019benchmarking,zhang_compact_2013}
to decrease the number of simulations needed, but it will still be of the
approximate order.

A different approach that has been gaining some popularity is
\emph{inverse design with adjoint simulation}.%
\cite{molesky_inverse_2018}
The idea is that if the gradient of the figure of merit
with respect to the parameters can be calculated, then we can use gradient based
optimization methods, which converge much faster, even if the number of
parameters is very large. With these methods, one hopes to be able to search
among a much more general class of designs for the optimal one.
This has been successfully applied to nanophotonic devices,
where a wide variety of components have been designed\cite{spins2019}.
In some cases, for example the waveguide bend, the inverse-designed device could
be made much more compact than conventionally designed
bends.\cite{jensen_systematic_2004}
In other cases, for example the vertically-incident wavelength-demultiplexing
grating coupler, there are no other known conventional methods of designing the
device.\cite{piggott_inverse_2014}

Quantum acoustic devices have in general been studied much less than photonic
devices, and the library of known devices is very small.
With this thesis, I explore the possibility of extending the inverse-design paradigm to
quantum acoustic devices.
Since both acoustics and electromagnetics are wave phenomena, there are many
analogies to be drawn, but there are also important differences.
I have in this work shown that the adjoint method is applicable to the case of
acoustics, as well as derived the form of the equations.
As a proof of concept, I attempt to design a phononic 50/50 beamsplitter.
The beamsplitter is conceptually one of the simplest devices imaginable,
and photonic beamsplitters have been studied in great detail for many years.
However, there is still no standard implementation for phononic beamsplitters.

\section{Aim and Thesis outline}

The aims of this thesis are:
\begin{itemize}
	\item Rederive the equations for inverse design with adjoint simulation in
		the case of acoustics and confirm that the methods are theoretically
		applicable.
	\item Implement the methods and use them to design a phononic beamsplitter.
\end{itemize}

Chapter 2 presents some of the theory on solid mechanics and acoustic waves needed to
understand the thesis.
A band diagram over the modes of the waveguide used is also shown there.
In chapter 3, the inverse design process is presented. First, the general
adjoint method is showcased, followed by a derivation in the specific case of
acoustics.
Then, gradient based optimization algorithms are discussed and the ones used in
this thesis are presented.
Chapter 4 describes the specifics of the simulations performed in this work:
both a specification of the skeleton of the devices, how the input wave was excited,
and how the designs were parametrized.

