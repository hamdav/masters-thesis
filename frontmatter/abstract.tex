\thesisImprintTitle:\\
\thesisImprintSubtitle\\[1ex]
\thesisAuthor\\
\thesisDepartment\\
\thesisUniversity

\thispagestyle{plain}           % Suppress header
\section*{\abstractname}
Phononic devices could enable and improve a broad range of functions in
the realm of classical and quantum information processing.
However, such devices are often designed through analytic methods, combined with brute-force
parameter sweeps, which severely limits the designs that can be investigated.
This work presents a method for inverse-design of traveling-wave phononic devices,
allowing a vastly larger design space to be explored.
At the heart of the method lies a fast calculation of gradients using the
adjoint method, whereby the gradient computation costs no more than a single
simulation.
I show that this method is theoretically applicable to phononic devices,
and demonstrate that it works in simulation.
As a proof-of-concept, I attempt to design a phononic beamsplitter.
The design process consists of two steps:
one with continuously varying materials which is non-physical,
but easier to optimize;
and one with binary devices, accomplished through level-set methods.
The first step yields near perfect performance, achieving less than one percent
reflection and almost nothing scattering into other modes.
The second step never reached as good performance, though still a 46/46
split was obtained with around 8~\% of the power reflected.
Though the resulting designs have some problems with small features and
sensitive performance,
this method looks promising for use in the design of future phononic devices.

\if\thesisType M
    \textbf{Keywords:}
\else
    \textbf{Nyckelord:}
\fi
\thesisKeywords.

% NOTE: this needs modification if the abstract is longer than one page
% (which it shouldn't be)
\if\thesisLayout 2
\newpage                % Create blank page
\thispagestyle{empty}
\mbox{}
\fi
