\thesisImprintTitle:\\
\thesisImprintSubtitle\\[1ex]
\thesisAuthor\\
\thesisDepartment\\
\thesisUniversity

\thispagestyle{plain}           % Suppress header
\section*{\abstractname}
Phononic devices could enable and improve a broad range of functions in
the realm of quantum computing and sensing as well as classical devices.
However, such devices are currently often designed by hand, combined with brute force
parameter sweeps, which severely limits the designs that can be investigated.
This work presents a method for inverse-design of phononic devices.
At the heart of the method lies a fast calculation of gradients using the
adjoint method.
I show that this method is theoretically applicable to phononic devices.
As a proof of concept, I also attempt to design a phononic beamsplitter using
this method.
The designed devices perform decently, splitting the input power 40/40 with the rest
being scattered into other modes or getting reflected.
The final designs depend very much on the initial design,
though the performance does not.
This method looks promising for use in the design of future phononic devices.

\if\thesisType M
    \textbf{Keywords:}
\else
    \textbf{Nyckelord:}
\fi
\thesisKeywords.

% NOTE: this needs modification if the abstract is longer than one page
% (which it shouldn't be)
\if\thesisLayout 2
\newpage                % Create blank page
\thispagestyle{empty}
\mbox{}
\fi
